\documentclass{article}
\usepackage[utf8]{inputenc}
\usepackage[a4paper,margin=1in]{geometry}
\usepackage{hyperref}
\title{OCR Pipeline Change Log}
\date{}
\begin{document}

\maketitle

\section*{Unreleased}
\begin{itemize}
  \item[2025-04-26 22:39] \textbf{Add \texttt{convert\_images\_to\_pdf} method} (initial step)  % commit: feat: add convert_images_to_pdf
  \begin{itemize}
    \item \textbf{What it does:} Scans the specified image directory, opens each file with PIL, converts to RGB, and saves as a PDF in an in-memory buffer.
    \item \textbf{Why it’s necessary:} Standardizes inputs as PDFs without writing to disk, simplifying downstream PDF-based processing and automatic cleanup.
  \end{itemize}

  \item[2025-04-26 22:39] \textbf{Add \texttt{render\_pdfs\_to\_images} method} (render PDFs back to images)  % commit: feat: add render_pdfs_to_images
  \begin{itemize}
    \item \textbf{What it does:} Uses \texttt{pdf2image.convert\_from\_bytes} to render each in-memory PDF buffer at 300 DPI, storing the first page as a PIL image.
    \item \textbf{Why it’s necessary:} Ensures a uniform, high-resolution image representation for cropping and OCR, bridging the PDF buffers to image processing.
  \end{itemize}

  \item[2025-04-26 22:39] \textbf{Add \texttt{crop\_regions} method} (crop page images into Symbol, Value1, Value2)  % commit: feat: add crop_regions step
  \begin{itemize}
    \item \textbf{What it does:} Computes column boundaries at 12\%, 52\%, and 100\% of the width on each rendered image and crops out three regions labeled Symbol, Value1, and Value2.
    \item \textbf{Why it’s necessary:} Isolates individual table fields to boost OCR accuracy by narrowing the recognition area to each specific column.
  \end{itemize}

  \item[2025-04-26 22:39] \textbf{Initialize LaTeX change log file}  % commit: docs: add OCRPIPELINE_CHANGELOG.tex
  \begin{itemize}
    \item \textbf{What it does:} Introduces a dedicated LaTeX document that mirrors this Markdown changelog.
    \item \textbf{Why it’s necessary:} Fulfills the professor’s requirement for formal documentation in LaTeX, yielding a cleanly typeset PDF for review.
  \end{itemize}
\end{itemize}

\end{document}
