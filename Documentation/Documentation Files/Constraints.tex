% constraints.tex – Project Constraints Document
\documentclass[a4paper,12pt]{article}
\usepackage[utf8]{inputenc}
\usepackage[T1]{fontenc}
\usepackage{enumitem}
\usepackage{geometry}
\geometry{margin=1in}
\title{Project Constraints}
\date{\today}

\begin{document}
\maketitle

\section*{Overview}
This document outlines the primary constraints encountered during the development
of our computer vision pipeline for extracting information from gear-drawing blueprints.

\section*{Constraints}
\begin{enumerate}[left=0pt]
\item \textbf{Blueprint Quality Constraints}
\begin{itemize}
\item Variability in image resolution, contrast, and noise levels can impair detection.
\item Missing or faint edges and non-uniform lighting can lead to incorrect feature extraction.
\end{itemize}

\item \textbf{Model Hallucination Risks}
\begin{itemize}
\item Absence of explicit bounding boxes around some parts may cause the model to hallucinate information.
\item Overlapping or poorly-defined regions can trigger false positives in symbol or value detection.
\end{itemize}

\item \textbf{Annotation Effort Constraints}
\begin{itemize}
\item Manual annotation of large crops (extracting the full part) is labor-intensive.
\item Manual annotation of small crops (detecting bounding boxes around embedded information) doubles the effort.
\item Ensuring consistency across annotators and iterations significantly increases time requirements.
\end{itemize}

\item \textbf{Computational Power Constraints}
\begin{itemize}
\item Training deep-learning models requires GPU resources, which may not always be readily available.
\item Limited access to high-performance hardware can extend training and iteration cycles.
\end{itemize}
\end{enumerate}

\end{document}