\documentclass{article}
\usepackage[utf8]{inputenc}
\usepackage{geometry}
\usepackage{longtable}
\usepackage{array}
\geometry{a4paper, margin=1in}

\begin{document}

\section*{Project Goals – Feasibility Analysis: Object Detection}

\begin{longtable}{|>{\centering\arraybackslash}p{0.8cm}|p{7cm}|p{7cm}|}
\hline
\textbf{N°} & \textbf{Target Description} & \textbf{Success Criteria (Measurability)} \\
\hline
Z1 & Automate evaluation of mechanical drawings to reduce manual checks. & YOLO model detects relevant drawing area (bounding boxes). Manual check no longer required for initial analysis. \\
\hline
Z2 & Detect the rectangle that contains technical info like values, tolerances, and part IDs. & Model identifies and localizes rectangles with >90\% accuracy in test set. \\
\hline
Z3 & Use OCR to extract relevant text/numbers from the detected rectangles. & OCR extracts text with >85\% accuracy and maps to structured format (e.g., JSON). \\
\hline
Z4 & Identify critical tolerances and feasibility issues based on extracted data and predefined rules. & System highlights values outside manufacturing specs automatically. Flagging works on >90\% of cases. \\
\hline
Z5 & Re-evaluate updated drawings automatically and highlight changes. & Updated parts are automatically compared to previous versions; changes are clearly marked. \\
\hline
Z6 & Develop a user interface for reviewing detected data and manual corrections if needed. & Web interface displays detected values and allows edits. 100\% of processed files are accessible via UI. \\
\hline
Z7 & Log system decisions and create a summary report for each drawing processed. & Each processed drawing has a log file with extracted data, tolerances, and final feasibility decision. \\
\hline
\end{longtable}

\end{document}
