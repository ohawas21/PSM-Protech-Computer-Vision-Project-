\documentclass[12pt]{article}
\usepackage[utf8]{inputenc}
\usepackage{geometry}
\geometry{margin=1in}
\usepackage{graphicx}
\usepackage{amsmath}
\usepackage{enumitem}
\usepackage{hyperref}

\title{Chapter 1: Initial Situation}
\author{AAI Project Team -- TH Rosenheim}
\date{\today}

\begin{document}

\maketitle

\section{Initial Situation}

The company \textbf{PSM-Protech} specializes in producing high-precision mechanical components and systems that demand accurate documentation and feasibility validation. A key challenge in their current workflow is the handling of large quantities of scanned technical documents, such as feasibility studies and product specifications, which are predominantly available as unstructured image-based PDFs.

At present, relevant numerical data—such as performance values, tolerances, and dimensions—are manually extracted from these documents by engineers and sales personnel. This manual approach is inefficient, time-consuming, and error-prone, especially when processing multiple documents or when fast turnaround is required for project quotes or internal evaluations.

Moreover, due to varied document formats, inconsistent layouts, and the presence of both typed and handwritten text, traditional parsing methods are ineffective. This further exacerbates operational inefficiencies and data inconsistencies across departments such as sales engineering, feasibility analysis, and quality assurance.

To address these pain points, the project aims to develop a hybrid AI-powered pipeline that includes:
\begin{itemize}
    \item an \textbf{OCR-based extraction component} for retrieving structured numerical data from scanned documents,
    \item an \textbf{image classification component} to support decision-making and automate visual assessments relevant to feasibility evaluation.
\end{itemize}

This system will streamline internal processes, reduce manual workload, and improve data accuracy and accessibility. Together, these components form the foundation of a smarter, semi-automated feasibility study workflow for PSM-Protech.

This project was initiated by a team of six students from the \textit{Agile AI (AAI)} course at \textbf{TH Rosenheim}, in collaboration with PSM-Protech. A kickoff meeting was held at the company's premises on \textbf{March 31st}, during which the project context, challenges, and goals were discussed in detail. The meeting also included a tour of the company's facilities to better understand the environment in which the solution would be deployed.

Following the meeting, both parties agreed to proceed with the project once a sample dataset was provided. However, as of today (\textbf{May 3rd}), the team has not yet received the promised dataset. This delay has affected the start of data-driven experimentation and model prototyping, which were planned for the early project phase.

Despite this challenge, preliminary work on the OCR and table extraction pipeline has begun based on synthetic and public test data, and the team remains committed to delivering a solution aligned with the company’s needs.

\end{document}

