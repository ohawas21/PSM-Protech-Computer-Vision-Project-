% Documentation/technologies.tex – Technologies Used
\documentclass[a4paper,12pt]{article}
\usepackage[utf8]{inputenc}
\usepackage[T1]{fontenc}
\usepackage{geometry}
\usepackage{array}
\geometry{margin=1in}
\title{Technologies Used}
\author{Your Name / Project Team}
\date{\today}

\begin{document}
\maketitle

\section*{Technologies and Purpose}

\begin{table}[h!]
  \centering
  % Simplified table using standard \hline to avoid booktabs midrule issues
  \begin{tabular}{|>{\raggedright}p{0.3\textwidth}|>{\raggedright}p{0.6\textwidth}|}
    \hline
    \textbf{Technology} & \textbf{Usage} \\
    \hline
    Object Detection (YOLOv8) & Precise detection of gear symbols and annotation regions within blueprint images. \\
    Classification Model & Categorization of extracted symbols into predefined classes for downstream processing. \\
    Model Training & Implementation of optimized training pipelines leveraging GPU resources for faster convergence. \\
    Python \& OpenCV & Scripting and image pre-/post-processing tasks, including cropping and augmentations. \\
    PyTorch / Ultralytics & Framework for defining, fine-tuning, and running deep-learning models for detection and classification. \\
    Docker \& Docker Compose & Containerization of services and environment reproducibility across development and deployment. \\
    GitLab CI/CD & Automated build, testing, and deployment workflows for continuous integration and delivery. \\
    Data Preprocessing (pandas, NumPy) & Tabular data manipulation, feature extraction, and numerical operations on annotation metadata. \\
    Visualization (Matplotlib, Seaborn) & Plotting training metrics, data distributions, and performance insights. \\
    Roboflow & Image labeling, dataset versioning, and annotation management for both synthetic and real datasets. \\
    \hline
  \end{tabular}
  \caption{Summary of technologies and their roles within the project.}
  \label{tab:technologies}
\end{table}

\end{document}
