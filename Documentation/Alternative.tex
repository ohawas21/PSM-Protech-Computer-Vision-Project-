% alternatives.tex – Alternative Approach Document
\documentclass\[a4paper,12pt]{article}
\usepackage\[utf8]{inputenc}
\usepackage\[T1]{fontenc}
\usepackage{geometry}
\usepackage{graphicx}
\geometry{margin=1in}
\title{Alternative Approach}
\author{Your Name / Project Team}
\date{\today}

\begin{document}
\maketitle

\section\*{Alternative Approach}
Due to delays in receiving customer-provided blueprints, we adopted an alternative workflow to ensure continuous progress:

\begin{enumerate}
\item \textbf{Synthetic Data Source:} We obtained sample drawings in the DIN file format containing demonstration gear diagrams.
\item \textbf{Bounding-Box Extraction:} From these demonstration files, we programmatically extracted bounding boxes around parts and their associated annotation regions.
\item \textbf{Model Prototyping:}
\begin{itemize}
\item \textbf{Classification Model:} Trained on synthetic crops of extracted symbols to assign them to predefined classes.
\item \textbf{OCR Extraction Model:} Trained on synthetic text crops to recognize numeric values and tolerances.
\end{itemize}
\item \textbf{Preliminary Results:} Both models achieved near-perfect performance (100% accuracy) on the synthetic dataset, validating our training pipeline.
\item \textbf{Integration with Real Data:} Upon delivery of actual customer blueprints, we plan to fine-tune both models using real annotations to account for domain differences.
\end{enumerate}

This alternative approach allowed us to parallelize model development and significantly short-circuit initial training cycles while waiting for production data.

\section\*{Reference Table: Geometric Tolerancing Specifications}
The following table was used as the primary reference for generating synthetic training data:

\begin{figure}\[h!]
\centering
\includegraphics\[width=\textwidth]{795f2e3e-f8f7-41aa-a8cc-bdb675be6122.png}
\caption{Main geometric tolerancing table used for synthetic data creation.}
\label{fig\:geom\_tol}
\end{figure}

\end{document}
