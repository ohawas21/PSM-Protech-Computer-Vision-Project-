% Commit 1: Add glossary entries A–C
\section*{Glossary (A–C)}
\begin{description}
  \item[AI (Artificial Intelligence)] 
    A broad field of computer science focused on creating software or machines that 
    perform tasks requiring human-like intelligence, such as decision-making, 
    pattern recognition, and learning from data.
  \item[API (Application Programming Interface)] 
    A set of rules and protocols that allow different software applications to 
    communicate and interact with each other.
  \item[Camelot] 
    An open-source Python library for automatically detecting and extracting 
    tables from PDF documents into structured data formats.
  \item[CI/CD (Continuous Integration/Continuous Deployment)] 
    Practices that automate the building, testing, and deployment of code changes, 
    allowing teams to integrate and deliver updates frequently and reliably.
  \item[Computer Vision] 
    A field of AI that enables computers to interpret and understand visual  
    information from images or videos, such as detecting objects or reading text.
\end{description}
\section*{Glossary (C–D)}
\begin{description}
  \item[Convolutional Neural Network (CNN)] 
    A deep learning model with layered filters designed to automatically learn and 
    recognize visual features in images, commonly used for tasks like classification 
    and object detection.
  \item[Containerization] 
    The process of packaging an application and its dependencies into a standardized 
    unit (container) that can run consistently across different computing environments.
  \item[CSV (Comma-Separated Values)] 
    A simple text file format where each line represents a row, and values in a row 
    are separated by commas; commonly used for storing and exchanging tabular data.
  \item[Deep Learning] 
    A subset of machine learning that uses multi-layered neural networks to learn 
    complex patterns from large datasets, enabling advanced tasks like image 
    recognition and language understanding.
  \item[DevOps] 
    A set of practices combining software development and IT operations aimed at 
    shortening development lifecycles and providing continuous delivery with high 
    software quality.
\end{description}


