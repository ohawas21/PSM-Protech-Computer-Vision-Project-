\section*{Glossary (A–C)}
\begin{description}
  \item[AI (Artificial Intelligence)] 
    A broad field of computer science focused on creating software or machines that 
    perform tasks requiring human-like intelligence, such as decision-making, 
    pattern recognition, and learning from data.
  \item[API (Application Programming Interface)] 
    A set of rules and protocols that allow different software applications to 
    communicate and interact with each other.
  \item[Camelot] 
    An open-source Python library for automatically detecting and extracting 
    tables from PDF documents into structured data formats.
  \item[CI/CD (Continuous Integration/Continuous Deployment)] 
    Practices that automate the building, testing, and deployment of code changes, 
    allowing teams to integrate and deliver updates frequently and reliably.
  \item[Computer Vision] 
    A field of AI that enables computers to interpret and understand visual  
    information from images or videos, such as detecting objects or reading text.
\end{description}
\section*{Glossary (C–D)}
\begin{description}
  \item[Convolutional Neural Network (CNN)] 
    A deep learning model with layered filters designed to automatically learn and 
    recognize visual features in images, commonly used for tasks like classification 
    and object detection.
  \item[Containerization] 
    The process of packaging an application and its dependencies into a standardized 
    unit (container) that can run consistently across different computing environments.
  \item[CSV (Comma-Separated Values)] 
    A simple text file format where each line represents a row, and values in a row 
    are separated by commas; commonly used for storing and exchanging tabular data.
  \item[Deep Learning] 
    A subset of machine learning that uses multi-layered neural networks to learn 
    complex patterns from large datasets, enabling advanced tasks like image 
    recognition and language understanding.
  \item[DevOps] 
    A set of practices combining software development and IT operations aimed at 
    shortening development lifecycles and providing continuous delivery with high 
    software quality.
\end{description}
\section*{Glossary (G–K)}
\begin{description}
  \item[Gaussian Blur] 
    An image preprocessing technique applying a Gaussian function to an image to 
    reduce noise and detail, often used before OCR to improve accuracy.
  \item[GitLab] 
    A web-based DevOps platform for source code management using Git, offering 
    repository hosting, issue tracking, code review, and built-in CI/CD pipelines.
  \item[GitLab CI/CD] 
    GitLab’s integrated continuous integration and continuous deployment system, 
    which runs automated pipelines (build, test, deploy) defined in a project’s repository.
  \item[Jupyter Notebook] 
    An interactive computing environment that allows users to combine live code, 
    narrative text, visualizations, and other media in a single document for exploration and sharing.
  \item[Kubernetes] 
    An open-source system for automating the deployment, scaling, and management of 
    containerized applications across clusters of machines.
\end{description}
\section*{Glossary (L–O)}
\begin{description}
  \item[LLM (Large Language Model)] 
    A type of deep learning model trained on vast text datasets that can generate 
    and understand human-like text, used in tasks like translation, summarization, and question-answering.
  \item[Logging] 
    The practice of recording runtime events, errors, and informational messages 
    from software into log files or consoles to help with debugging and monitoring.
  \item[Nomad] 
    An open-source scheduler and orchestrator for deploying and managing both containerized and non-containerized applications across a cluster.
  \item[NumPy] 
    A core Python library for numerical computing, providing efficient operations on 
    large, multi-dimensional arrays and matrices.
  \item[OCR (Optical Character Recognition)] 
    Technology that converts images of text into editable digital text, enabling  
    computers to “read” scanned documents or photos of text.
\end{description}
\section*{Glossary (O–P)}
\begin{description}
  \item[OpenCV] 
    An open-source library for image processing and computer vision tasks, offering 
    functions to transform, analyze, and manipulate images.
  \item[pandas] 
    A high-level Python library for data manipulation and analysis, providing DataFrame 
    structures similar to spreadsheets for handling tabular data.
  \item[Pathlib] 
    A Python standard library module for object-oriented filesystem path operations, 
    providing classes and methods to handle filesystem paths.
  \item[pdf2image] 
    A Python library that converts PDF pages into PIL image objects, enabling image 
    processing on PDF content.
  \item[PDF (Portable Document Format)] 
    A file format developed by Adobe for presenting documents in a manner independent 
    of application software, hardware, and operating systems.
\end{description}

