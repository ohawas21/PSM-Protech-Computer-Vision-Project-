% related_work.tex – Related Work Documentation
\documentclass[a4paper,12pt]{article}
\usepackage[utf8]{inputenc}
\usepackage[T1]{fontenc}
\usepackage{geometry}
\usepackage{enumitem}
\geometry{margin=1in}
\title{Related Work}
\author{Your Name / Project Team}
\date{\today}

\begin{document}
\maketitle

\section*{Overview}
This section reviews existing approaches and tools for extracting structured data from engineering
and technical drawings, highlighting methods relevant to our project.

\section*{Academic Approaches}
\begin{itemize}[left=0pt]
\item \textbf{Hybrid Deep-Learning for Structured Extraction}: Zheng et al. integrate an oriented
bounding-box detector (YOLOv11) with a document-understanding transformer (Donut) to
extract and parse key annotation categories (e.g., GD&T, tolerances, materials).
\item \textbf{Tolerance Information Extraction}: Xu et al. (2024) combine image-processing pipelines
with a CNN-based recognizer to extract tolerance data from mechanical drawings.
\item \textbf{eDOCr2 Framework}: The eDOCr2 framework (MDPI, 2023) fuses conventional OCR with
tailored image-processing to pull structured data (dimensions, notes, tolerances) from 2D
technical drawings.
\end{itemize}

\section*{Open-Source and Commercial Solutions}
\begin{itemize}[left=0pt]
\item \textbf{engineering-drawing-extractor (GitHub)}: Bakkopi’s project automates region isolation
and OCR to capture drawing metadata (numbers, titles, authors).
\item \textbf{Werk24 API}: A commercial AI service offering specialized extraction of structured
data (including tolerances and BOM) from technical drawings.
\item \textbf{Infrrd and Markovate AI Blueprint Reader}: Platforms combining machine learning,
OCR, and workflow automation to streamline blueprint processing in preconstruction planning.
\end{itemize}

\section*{Relevance to Our Approach}
Our method of bootstrapping with synthetic DIN-format data and advancing with YOLO-based detection,
classification, and OCR aligns with these hybrid detection-and-parsing paradigms, while uniquely
addressing data availability delays.
\end{document}