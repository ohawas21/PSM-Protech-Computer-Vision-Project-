\documentclass{article}
\usepackage[utf8]{inputenc}
\usepackage{longtable}
\usepackage{geometry}
\geometry{margin=1in}

\title{Project Milestones and Timeline}
\author{Team Wunderbar}
\date{\today}

\begin{document}

\maketitle

\section*{Weekly Project Timeline and Milestones}
\section*{Weekly Project Timeline and Milestone ups}

\begin{longtable}{|p{2.5cm}|p{4.2cm}|p{8.5cm}|}
\hline
\textbf{Week} & \textbf{Dates (Mon–Sun)} & \textbf{Milestones and Tasks} \\
\hline
Week 1 & April 7 -- April 13, 2025 &
\begin{itemize}
    \item Set up the project repository.
    \item Define folder structure and annotation format.
    \item Finalize task board and project management workflow.
    \item Conduct kickoff meeting to align goals and responsibilities.
\end{itemize} \\
\hline
Week 2 & April 14 -- April 20, 2025 &
\begin{itemize}
    \item Begin Phase 1 annotation on original images for Model 1 (YOLO).
    \item Set up initial YOLO environment and baseline configuration.
\end{itemize} \\
\hline
Week 3 & April 21 -- April 27, 2025 &
\begin{itemize}
    \item Continue Phase 1 annotation.
    \item Train Model 1 (YOLO) with partial annotated data.
\end{itemize} \\
\hline
Week 4 & April 28 -- May 4, 2025 &
\begin{itemize}
    \item Finalize YOLO training and bounding box outputs.
    \item Generate cropped outputs from detected regions.
\end{itemize} \\
\hline
Week 5 & May 5 -- May 11, 2025 &
\begin{itemize}
    \item Start Phase 2 annotation on cropped regions for symbol classification.
    \item Begin training Model 2 to classify domain-specific symbols.
\end{itemize} \\
\hline
Week 6 & May 12 -- May 18, 2025 &
\begin{itemize}
    \item Continue symbol classification annotation.
    \item Improve Model 2 accuracy and finalize training.
\end{itemize} \\
\hline
Week 7 & May 19 -- May 25, 2025 &
\begin{itemize}
    \item Begin Phase 3 OCR annotation on cropped image regions.
    \item Train Model 3 for text and ID extraction.
\end{itemize} \\
\hline
Week 8 & May 26 -- June 1, 2025 &
\begin{itemize}
    \item Finalize OCR model (Model 3) and test integration with symbol output.
    \item Begin rule and formula development for feasibility calculations.
\end{itemize} \\
\hline
Week 9 & June 2 -- June 8, 2025 &
\begin{itemize}
    \item Integrate all models (YOLO, symbol, OCR, calculation).
    \item Develop Flask frontend interface.
    \item Set up Docker environment for deployment.
\end{itemize} \\
\hline
Week 10 & June 9 -- June 15, 2025 &
\begin{itemize}
    \item Final QA for pipeline and frontend.
    \item Deploy complete application using Docker.
    \item Validate calculations and usability through interface.
    \item Submit project presentation and deliverables.
\end{itemize} \\
\hline
\end{longtable}

\section*{Annotation and Model Workflow}

\subsection*{Model 1: Object Detection (YOLO)}
\begin{itemize}
    \item Annotate 300--400 images for key object regions.
    \item Train YOLO model to detect target zones from input images.
    \item Output includes bounding boxes used for further processing.
\end{itemize}

\subsection*{Model 2: Symbol Classification}
\begin{itemize}
    \item Use YOLO output to crop relevant zones.
    \item Annotate and classify symbols (arrows, signs, icons, etc.).
    \item Train a symbol classification model to label cropped regions.
\end{itemize}

\subsection*{Model 3: OCR (Text Recognition)}
\begin{itemize}
    \item Annotate cropped zones for text extraction.
    \item Train OCR model to extract IDs, codes, or structured text.
    \item Combine outputs with symbol classifications.
\end{itemize}

\subsection*{Model 4: Rule-Based Feasibility Analysis}
\begin{itemize}
    \item Use outputs from previous models as input.
    \item Apply domain-specific formulas and rule logic.
    \item Derive feasibility values and interpretations.
    \item Export results to Excel/CSV reports.
\end{itemize}

\end{document}