\documentclass[a4paper,12pt]{article}
\usepackage[utf8]{inputenc}
\usepackage{amsmath}
\usepackage{enumitem}
\usepackage{geometry}
\usepackage{titlesec}
\usepackage{graphicx}
\usepackage{caption}

\geometry{margin=1in}
\titleformat{\section}{\normalfont\Large\bfseries}{\thesection}{1em}{}
\titleformat{\subsection}{\normalfont\large\bfseries}{\thesubsection}{1em}{}

\begin{document}

\section*{Notes to Entry}

\textbf{Note 3 to entry:} A TED can be explicit or implicit. When indicated, an explicit TED is indicated by a rectangular frame including a value and sometimes an associated symbol, e.g. $\emptyset$ or R. On 3D models, explicit TEDs may be available by queries.

\textbf{Note 4 to entry:} An implicit TED is not indicated. An implicit TED is one of the following: 0 mm, $0^\circ$, $90^\circ$, $180^\circ$, $270^\circ$ and the angular distance between equally spaced features on a complete circle.

\textbf{Note 5 to entry:} TEDs are not affected by individual or general specifications.

\section*{3.8 Theoretically Exact Feature (TEF)}

Nominal feature with ideal shape, size, orientation and location, as applicable.

\textbf{Note 1 to entry:} A theoretically exact feature (TEF) can have any shape and can be defined by explicitly indicated theoretically exact dimensions (TEDs) or implicitly defined in CAD data.

\textbf{Note 2 to entry:} The theoretically exact location and orientation, if applicable, is relative to the indicated datum system for the specification of the corresponding actual feature.

\textbf{Note 3 to entry:} See also ISO 25378.

\textbf{Example 1:} The spherical surface shown in Figure 110 is a theoretically exact feature, with a defined spherical radius and a defined location and orientation relative to datum A.

\textbf{Example 2:} A virtual condition, e.g. a maximum material virtual condition (MMVC) according to ISO 2692, is a theoretically exact feature.

\section*{3.9 United Feature}

Compound integral feature which may or may not be continuous, considered as a single feature.

\textbf{Note 1 to entry:} A united feature can have a derived feature.

\textbf{Note 2 to entry:} The definition of a united feature is intentionally very broad to avoid excluding any useful applications. However, it is not intended that a united feature can be used to define something that is by nature several separate features. For example, building a united feature from two parallel, non-coaxial cylindrical features, or two parallel, non-coaxial rectangular tubes (each built from two perpendicular pairs of parallel planes) is not an intended use.

\textbf{Example 1:} A cylindrical feature defined from a set of arc features, such as the outside diameter of a spline, is an intended use of a united feature, see Figure 48.

\textbf{Example 2:} Two complete coaxial cylinders, which do not have the same nominal diameter, cannot be considered as a united feature.

\section*{4 Basic Concepts}

\subsection*{4.1}
Geometrical tolerances shall be specified in accordance with functional requirements. Manufacturing and inspection requirements can also influence geometrical tolerancing.

\textbf{Note:} Indicating geometrical tolerances does not necessarily imply the use of any particular method of production, measurement or gauging.

\subsection*{4.2}
A geometrical tolerance applied to a feature defines the tolerance zone around the reference feature within which the toleranced feature shall be contained.

\textbf{Note 1:} In some cases, i.e. when using the characteristic parameter modifiers introduced in this document, see Figure 13, geometrical specifications can define characteristics instead of zones.

\textbf{Note 2:} All dimensions given in the figures in this document are in millimetres.

\subsection*{4.3}
A feature is a specific portion of the workpiece, such as a point, a line or a surface; these features can be integral features (e.g. the external surface of a cylinder) or derived features (e.g. a median line or median surface). See ISO 17450-1.

\subsection*{4.4}
Depending on the characteristic to be specified and the manner in which it is specified, the tolerance zone is one of the following:

\begin{itemize}[label=--]
    \item the space within a circle;
    \item the space between two concentric circles;
    \item the space between two parallel circles on a conical surface;
    \item the space between two parallel circles of the same diameter;
    \item the space between two equidistant complex lines or two parallel straight lines;
    \item the space between two non-equidistant complex lines or two non-parallel straight lines;
    \item the space within a cylinder;
    \item the space between two coaxial cylinders;
    \item the space within a cone;
    \item the space within a single complex surface;
    \item the space between two equidistant complex surfaces or two parallel planes;
    \item the space within a sphere;
    \item the space between two non-equidistant complex surfaces or two non-parallel planes.
\end{itemize}

\textbf{Note:} The tolerance zone may be defined in the CAD model.

\subsection*{4.5}
Unless a more restrictive indication is required, for example by an explanatory note, the toleranced feature may be of any form, orientation and/or location within this tolerance zone.

\subsection*{4.6}
The specification applies to the whole extent of the considered feature unless otherwise specified. See Clauses 11 and 12. Currently, the detailed rules for partitioning (defining the boundary of the toleranced feature) are not elaborated in GPS standards. This leads to an ambiguity of specification.

\subsection*{4.7}
Geometrical specifications which are assigned to features related to a datum(s) do not limit the form deviations of the datum feature(s) itself.

\subsection*{4.8}
For functional reasons, one or more characteristics can be specified to define the geometrical deviations of a feature. Certain types of specifications, which limit the geometrical deviations of a toleranced feature, can also limit other types of deviations for the same feature.

\begin{itemize}[label=--]
    \item A location specification controls location deviation, orientation deviation and form deviation of the toleranced feature.
    \item An orientation specification controls orientation and form deviations of the toleranced feature but cannot control its location.
    \item A form specification controls only form deviations of the toleranced feature.
\end{itemize}

\end{document}
